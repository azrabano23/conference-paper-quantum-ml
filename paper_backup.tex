\documentclass[10pt,conference]{IEEEtran}


% =======================
% Packages (keep minimal)
% =======================
\usepackage{cite}
\usepackage{amsmath,amssymb}
\usepackage{graphicx}
\usepackage{xcolor}
\usepackage{multirow}


% IEEEtran-friendly graphics path (your repo figures are in repo root)
\graphicspath{{./}}


% =======================
% Title (shortened a bit)
% =======================
\title{Analyzing Images of Blood Cells with Quantum Machine Learning Methods: Equilibrium Propagation and Variational Quantum Circuits to Detect Acute Myeloid Leukemia}


% =======================
% Double-blind author block
% (Replace with real authors only AFTER acceptance / camera-ready)
% =======================
\begin{document}

\author{\IEEEauthorblockN{Azra Bano}
\IEEEauthorblockA
{\textit{Electrical and Computer Engineering}\\
\textit{Rutgers University} \\
Piscataway, NJ, USA \\
ab2895@scarletmail.rutgers.edu}

\and
\IEEEauthorblockN{Larry S. Liebovitch*}
\IEEEauthorblockA{\textit{AC4 in the Climate School} \\
\textit{Columbia University} \\
New York, NY, USA \\
lsl2140@columbia.edu}


}
\thanks{*Corresponding author: L. S. Liebovitch email:LSL2140@columbia.edu.}

\maketitle
\IEEEpeerreviewmaketitle
\author{\IEEEauthorblockN{Anonymous Authors}}

\maketitle


% =======================
% Abstract (tight + IEEE)
% =======================
\begin{abstract}
This paper presents a feasibility study testing Quantum Machine Learning (QML) algorithms on a real-world medical imaging task using standard laptop hardware rather than actual quantum computers. We evaluate Equilibrium Propagation (EP)—an energy-based learning method that avoids backpropagation, which is incompatible with Quantum systems since backpropagation requires measurements that collapse quantum states—and Variational Quantum Circuits (VQCs) for automated detection of Acute Myeloid Leukemia (AML) from blood cell microscopy images.

Using limited subsets (50--250 samples per class) of the AML-Cytomorphology dataset (18,365 expert-annotated images), we demonstrate what Quantum algorithms can accomplish despite constraints: reduced image resolution (64×64 pixels vs. original high-resolution), limited training samples, and classical simulation via Qiskit rather than quantum hardware. EP achieves 86.4\% accuracy with energy-based learning, while the simulated 4-qubit VQC attains 83.0\% accuracy with remarkable stability across dataset scales. These results establish a reproducible benchmark for Quantum-inspired and Quantum learning in medical imaging, validating feasibility for future deployment on actual quantum processors.
\end{abstract}


\begin{IEEEkeywords}
Quantum Machine Learning, Variational Quantum Circuits, Equilibrium Propagation, Quantum Neural Networks, Quantum Benchmarks, Medical Image Analysis
\end{IEEEkeywords}


% =======================
% Introduction (tight)
% =======================
\section{Introduction}
Automated microscopy-based analysis can support clinical workflows by reducing the time and variability associated with manual blood smear review. Deep convolutional networks have demonstrated highly accurate differentiation of bone marrow cell morphologies \cite{Matek2021}, achieving performance that often surpasses human-level accuracy \cite{LeCun2015}. However, these classical methods typically rely on backpropagation and require substantial computational resources, motivating exploration of alternative learning paradigms better aligned with emerging hardware architectures.

Quantum Machine Learning (QML) has gained significant momentum with the development of Noisy Intermediate-Scale Quantum (NISQ) devices and hybrid algorithms such as Variational Quantum Circuits (VQCs) \cite{Benedetti2019,Farhi2018}. VQCs encode classical data into quantum states via feature maps and optimize circuit parameters through classical optimization routines. While many QML demonstrations utilize toy datasets (MNIST, synthetic data), evaluating these methods on real clinical data with inherent variability and complexity remains critical for understanding practical applicability.

In parallel, Equilibrium Propagation (EP) \cite{Scellier2017} offers a Quantum-inspired energy-based learning algorithm that trains networks by computing updates from differences between equilibrium states under free and nudged dynamics. Crucially, EP avoids explicit backpropagation—an essential feature since backpropagation is fundamentally incompatible with Quantum systems: gradient computation in quantum circuits requires intermediate measurements that collapse quantum states, thereby destroying superposition and entanglement, which defeats the computational advantages of quantum parallelism. This incompatibility makes EP particularly attractive for Quantum-inspired neuromorphic hardware and physical relaxation-based systems.


This paper evaluates EP and VQC models on AML detection from blood cell microscopy images using the AML-Cytomorphology dataset \cite{Matek2019}. Our contributions are:
\begin{itemize}
\item A reproducible EP pipeline for AML detection using engineered image features, trained without backpropagation.
\item A 4-qubit VQC classifier using ZZFeatureMap encoding and a shallow RealAmplitudes ansatz.
\item A benchmark across dataset scales (50--250 samples/class) reporting accuracy and runtime for QML-oriented evaluation.
\end{itemize}


\textbf{Open Science Note:} upon acceptance of our paper, we will will make our data/code available online.


% =======================
% Methods (short + clear)
% =======================
\section{Methods}


\subsection{Dataset and Experimental Protocol}

\textbf{AML-Cytomorphology Dataset:} We utilize the publicly available AML-Cytomorphology dataset \cite{Matek2019}, accessible at \texttt{https://doi.org/10.7937/tcia.2019.36f5o9ld}. This dataset comprises 18,365 single-cell blood smear microscopy images from 200 patients: 100 diagnosed with Acute Myeloid Leukemia (AML) and 100 healthy controls. Images were acquired at Munich University Hospital using standardized May-Gr\"unwald-Giemsa staining protocols under 100× oil immersion magnification. Each image captures a centered leukocyte with expert annotations from board-certified hematologists following WHO diagnostic criteria \cite{Matek2021}.

The original dataset contains high-resolution images; however, to assess Quantum algorithm performance under practical computational constraints and to enable fair comparison with NISQ-scale quantum circuits, we resize images to 64×64 pixels. Notably, despite this significant resolution reduction, our methods achieve strong classification performance—demonstrating that Quantum Machine Learning algorithms can extract discriminative features even from compressed representations. This finding is particularly relevant for resource-constrained deployment scenarios and edge computing applications.

\textbf{Experimental Subsets:} To study data efficiency critical for Quantum algorithms with limited qubit resources, we construct four balanced datasets with 50, 100, 200, and 250 samples per class (100, 200, 400, and 500 total samples respectively). All experiments employ stratified 80/20 train-test splits with fixed random seeds to ensure reproducibility.

\textbf{Feature Engineering:} To reduce dimensionality suitable for 4-qubit circuits, EP and VQC operate on 20 engineered scalar features extracted from each image rather than raw pixels. For classical baseline comparison, we implement standard CNN and dense neural network architectures within the same experimental framework.


\subsection{Feature Extraction}
Each image is converted to grayscale and summarized by 20 engineered features:
(i) intensity statistics, (ii) GLCM texture descriptors, (iii) morphology metrics, (iv) edge density/variation, and (v) frequency-domain summary statistics (FFT magnitude).
This representation is consistent with practical constraints of small-qubit VQCs and enables controlled comparisons across learning paradigms.


\subsection{Equilibrium Propagation Model}
EP trains a layered energy-based network through two phases \cite{Scellier2017}. In the free phase, the network relaxes to an equilibrium state given an input. In the nudged phase, output units are weakly biased toward target labels; parameter updates are computed from the difference between the two equilibria. The EP architecture used here is a fully connected network with hidden sizes 256--128--64 and a 2-unit output layer. Training uses momentum updates and early stopping to avoid overfitting on small subsets.


\subsection{Variational Quantum Circuit Classifier}
The VQC is implemented with 4 qubits and evaluated using classical simulation via Qiskit 0.39.0 (\texttt{https://qiskit.org}), IBM's open-source Quantum computing framework. This simulation establishes performance baselines prior to deployment on actual NISQ hardware. Input features are reduced from 20 to 4 dimensions via Principal Component Analysis (PCA retaining 95\% variance) and rescaled to [0, 2$\pi$] to match rotation gate domains.

We employ a ZZFeatureMap for quantum data encoding, which creates entanglement between all qubit pairs through second-order Pauli-Z expansions, and a shallow hardware-efficient RealAmplitudes ansatz (2 layers, 8 trainable parameters). This shallow circuit design (depth 12) is optimized for NISQ devices with limited coherence times. Circuit parameters are optimized using the gradient-free COBYLA algorithm over 200 iterations. Classification employs an expectation-value measurement on the first qubit ($\langle Z_0 \rangle$), thresholded at zero for binary label assignment.


% =======================
% Results (compact)
% =======================
\section{Results}


\subsection{Accuracy and Runtime}
Table~\ref{tab:main} reports accuracy and runtime on 250 samples per class. EP reaches 86.4\% accuracy with 0.13s inference on the test batch, while the simulated 4-qubit VQC attains 83.0\% accuracy. Classical baselines provide reference points for accuracy and computational cost.


\begin{table}[t]
\centering
\caption{Performance on 250 samples per class. Runtimes reflect classical simulation on laptop hardware and do not represent actual quantum computer execution times.}
\label{tab:main}
\begin{tabular}{|l|c|c|c|}
\hline
\textbf{Method} & \textbf{Accuracy} & \textbf{Training} & \textbf{Test} \\
\hline
CNN (Classical) & 98.4\% & 745s & 0.19s \\
Dense NN (Classical) & 92.0\% & 0.47s & 0.001s \\
EP (Quantum-inspired) & 86.4\% & 89.4s & 0.13s \\
VQC (Quantum, Qiskit sim) & 83.0\% & 180s & 1.0s \\
\hline
\end{tabular}
\end{table}


\subsection{Scaling with Dataset Size}
Figure~\ref{fig:acc} summarizes accuracy across dataset scales (50--250 samples/class). EP improves with more data and shows moderate variance under sampling. The VQC exhibits comparatively stable accuracy across scales in simulation, motivating further study on real hardware with noise and shot-based estimation.


\begin{figure}[t]
\centering
\includegraphics[width=0.48\textwidth]{figure1_accuracy_comparison.png}
\caption{Accuracy across dataset sizes for classical baselines, EP, and VQC.}
\label{fig:acc}
\end{figure}


% Optional second figure if you have room:
% \begin{figure}[t]
% \centering
% \includegraphics[width=0.48\textwidth]{figure2_efficiency_analysis.png}
% \caption{Accuracy vs training time trade-offs.}
% \label{fig:eff}
% \end{figure}


% =======================
% Discussion (tight + careful)
% =======================
\section{Discussion}
EP provides a competitive accuracy--efficiency trade-off while avoiding explicit backpropagation, which may be advantageous for hardware-aware learning systems. The simulated VQC achieves comparable performance despite operating with only four qubits and a shallow circuit, suggesting that quantum feature encodings can be effective on compact representations of microscopy images.


Limitations include: (i) the VQC evaluation is simulation-based and does not capture device noise, (ii) feature engineering constrains the representational capacity compared to end-to-end CNNs, and (iii) the study focuses on binary AML vs. healthy classification rather than a full multi-class hematology taxonomy. Future work will evaluate shot-based sampling, noise-aware training, error mitigation, and hybrid models that combine classical feature extractors with quantum classifiers, as well as benchmarking on additional clinical datasets.


% =======================
% Conclusion (short)
% =======================
\section{Conclusion}
We benchmarked equilibrium propagation and a 4-qubit variational quantum classifier for AML detection from blood cell microscopy images. EP achieved up to 86.4\% accuracy with efficient inference, while the simulated VQC achieved 83.0\% accuracy with stable behavior across dataset scales. These results provide a practical benchmark for QML and quantum-inspired learning algorithms in medical image analysis and motivate follow-on studies on noisy quantum hardware.


% =======================
% Acknowledgments (REMOVE for double-blind)
% Keep this only for camera-ready.
% Also: IEEE requires disclosure if AI-generated text was used.
% =======================
% \section*{Acknowledgments}
% The authors thank ... (dataset providers, funding, etc.)
% AI disclosure example (edit to match your usage):
% Portions of Sections I--V were edited with assistance from ChatGPT (OpenAI) for language clarity; technical content, experiments, and conclusions are the authors' own.


% =======================
% Data and Code Availability
% =======================
\section*{Data and Code Availability}

\textbf{Dataset:} The AML-Cytomorphology dataset is publicly available at The Cancer Imaging Archive: \texttt{https://doi.org/10.7937/tcia.2019.36f5o9ld}

\textbf{Code:} All implementation code, trained models, and experimental scripts are available at: \texttt{https://github.com/azrabano23/quantum-blood-cell-classification}

This repository includes: (1) Equilibrium Propagation implementation (\texttt{equilibrium\_propagation.py}), (2) Variational Quantum Circuit classifier (\texttt{vqc\_classifier.py}), (3) Classical baselines (\texttt{classical\_cnn.py}, \texttt{classical\_dense\_nn.py}), (4) Feature extraction utilities, and (5) Experiment reproduction scripts with detailed documentation.

% =======================
% References (trim to 8–10)
% =======================
\begin{thebibliography}{00}


\bibitem{LeCun2015}
Y.~LeCun, Y.~Bengio, and G.~Hinton, ``Deep learning,'' \emph{Nature}, vol.~521, no.~7553, pp.~436--444, 2015.


\bibitem{Benedetti2019}
M.~Benedetti, E.~Lloyd, S.~Sack, and M.~Fiorentini, ``Parameterized quantum circuits as machine learning models,'' \emph{Quantum Science and Technology}, vol.~4, no.~4, p.~043001, 2019.


\bibitem{Farhi2018}
E.~Farhi and H.~Neven, ``Classification with quantum neural networks on near-term processors,'' arXiv:1802.06002, 2018.


\bibitem{Havlicek2019}
V.~Havl{\'\i}{\v{c}}ek \emph{et al.}, ``Supervised learning with quantum-enhanced feature spaces,'' \emph{Nature}, vol.~567, no.~7747, pp.~209--212, 2019.


\bibitem{Scellier2017}
B.~Scellier and Y.~Bengio, ``Equilibrium propagation: Bridging the gap between energy-based models and backpropagation,'' \emph{Frontiers in Computational Neuroscience}, vol.~11, p.~24, 2017.


\bibitem{Matek2019}
C.~Matek, S.~Schwarz, C.~Marr, and K.~Spiekermann, ``A single-cell morphological dataset of leukocytes from AML patients and non-malignant controls,'' \emph{The Cancer Imaging Archive}, 2019, doi:10.7937/tcia.2019.36f5o9ld.

\bibitem{Matek2021}
C.~Matek, S.~Schwarz, K.~Spiekermann, and C.~Marr, ``Highly accurate differentiation of bone marrow cell morphologies using deep neural networks on a large image dataset,'' \emph{Blood}, vol.~138, no.~20, pp.~1917--1927, 2021.


\bibitem{Kandala2017}
A.~Kandala \emph{et al.}, ``Hardware-efficient variational quantum eigensolver for small molecules and quantum magnets,'' \emph{Nature}, vol.~549, no.~7671, pp.~242--246, 2017.


\bibitem{Davies2018}
M.~Davies \emph{et al.}, ``Loihi: A neuromorphic manycore processor with on-chip learning,'' \emph{IEEE Micro}, vol.~38, no.~1, pp.~82--99, 2018.


\end{thebibliography}


\end{document}