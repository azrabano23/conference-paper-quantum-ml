\documentclass[10pt,conference]{IEEEtran}


% =======================
% Packages (keep minimal)
% =======================
\usepackage{cite}
\usepackage{amsmath,amssymb}
\usepackage{graphicx}
\usepackage{xcolor}
\usepackage{multirow}


% IEEEtran-friendly graphics path (your repo figures are in repo root)
\graphicspath{{./}}


% =======================
% Title (shortened a bit)
% =======================
\title{Analyzing Images of Blood Cells with Equilibrium Propagation and Variational Quantum Circuits to Detect Acute Myeloid Leukemia}


% =======================
% Double-blind author block
% (Replace with real authors only AFTER acceptance / camera-ready)
% =======================
\begin{document}

\author{\IEEEauthorblockN{Azra Bano}
\IEEEauthorblockA
{\textit{Electrical and Computer Engineering}\\
\textit{Rutgers University} \\
Piscataway, NJ, USA \\
ab2895@scarletmail.rutgers.edu}

\and
\IEEEauthorblockN{Larry S. Liebovitch*}
\IEEEauthorblockA{\textit{AC4 in the Climate School} \\
\textit{Columbia University} \\
New York, NY, USA \\
lsl2140@columbia.edu}


}
\thanks{*Corresponding author: L. S. Liebovitch email:LSL2140@columbia.edu.}

\maketitle
\IEEEpeerreviewmaketitle
\author{\IEEEauthorblockN{Anonymous Authors}}

\maketitle


% =======================
% Abstract (tight + IEEE)
% =======================
\begin{abstract}
Equilibrium propagation (EP) provides an energy-based learning rule that avoids explicit backpropagation and is therefore appealing for hardware-constrained learning systems. In parallel, variational quantum circuits (VQCs) are a leading approach for near-term quantum machine learning on noisy intermediate-scale quantum (NISQ) devices. This paper evaluates EP and VQC models on a clinically relevant binary classification task: automated detection of Acute Myeloid Leukemia (AML) from blood cell microscopy images.


Experiments are conducted using the AML-Cytomorphology dataset (18{,}365 expert-annotated single-cell images). To enable fair comparison under NISQ constraints, EP and VQC operate on engineered image features; the VQC uses a 4-qubit circuit with a ZZFeatureMap encoder and RealAmplitudes ansatz optimized by COBYLA. Across balanced subsets (50--250 samples per class), EP achieves up to 86.4\% accuracy with sub-second inference, while the simulated VQC attains 83.0\% accuracy with stable performance across dataset scales. Results quantify accuracy--efficiency tradeoffs relative to classical baselines and provide a reproducible benchmark for quantum-inspired and quantum learning pipelines in medical image analysis.
\end{abstract}


\begin{IEEEkeywords}
Quantum Machine Learning, Variational Quantum Circuits, Equilibrium Propagation, Quantum Neural Networks, Quantum Benchmarks, Medical Image Analysis
\end{IEEEkeywords}


% =======================
% Introduction (tight)
% =======================
\section{Introduction}
Automated microscopy-based analysis can support clinical workflows by reducing the time and variability associated with manual blood smear review. Deep convolutional networks can achieve strong performance on medical imaging tasks, on training that typically relies on backpropagation and substantial compute resources \cite{LeCun2015}. These constraints motivate alternative learning paradigms that may better align with emerging hardware.


Quantum machine learning (QML) has gained momentum with the development of NISQ devices and hybrid algorithms such as variational quantum circuits (VQCs) \cite{Benedetti2019,Farhi2018}. VQCs encode classical data into quantum states and optimize circuit parameters via classical routines. While many demonstrations use small benchmark datasets, evaluating QML methods on real clinical data remains important for understanding practical behavior under realistic data variability.


Separately, equilibrium propagation (EP) \cite{Scellier2017} is an energy-based learning algorithm that computes updates from differences between equilibrium states under free and weakly nudged dynamics. EP avoids explicit gradient backpropagation and offers an attractive training rule for hardware-aware learning, including neuromorphic and physical relaxation-based systems.


This paper evaluates EP and VQC models on AML detection from blood cell microscopy images using the AML-Cytomorphology dataset \cite{Matek2019}. Our contributions are:
\begin{itemize}
\item A reproducible EP pipeline for AML detection using engineered image features, trained without backpropagation.
\item A 4-qubit VQC classifier using ZZFeatureMap encoding and a shallow RealAmplitudes ansatz.
\item A benchmark across dataset scales (50--250 samples/class) reporting accuracy and runtime for QML-oriented evaluation.
\end{itemize}


\textbf{Open Science Note:} upon acceptance of our paper, we will will make our data/code available online.


% =======================
% Methods (short + clear)
% =======================
\section{Methods}


\subsection{Dataset and Experimental Protocol}
We use the AML-Cytomorphology dataset \cite{Matek2019}, containing 18{,}365 single-cell microscopy images from AML patients and non-malignant controls. To study data efficiency relevant to QML settings, we create balanced subsets with 50, 100, 200, and 250 samples per class. All methods use stratified 80/20 train-test splits with fixed seeds.


To reduce input dimensionality for NISQ-scale circuits, EP and VQC operate on engineered features extracted from each image (20-D). For classical reference, we also report results from standard baselines (CNN and dense network) implemented in the same experimental framework.


\subsection{Feature Extraction}
Each image is converted to grayscale and summarized by 20 engineered features:
(i) intensity statistics, (ii) GLCM texture descriptors, (iii) morphology metrics, (iv) edge density/variation, and (v) frequency-domain summary statistics (FFT magnitude).
This representation is consistent with practical constraints of small-qubit VQCs and enables controlled comparisons across learning paradigms.


\subsection{Equilibrium Propagation Model}
EP trains a layered energy-based network through two phases \cite{Scellier2017}. In the free phase, the network relaxes to an equilibrium state given an input. In the nudged phase, output units are weakly biased toward target labels; parameter updates are computed from the difference between the two equilibria. The EP architecture used here is a fully connected network with hidden sizes 256--128--64 and a 2-unit output layer. Training uses momentum updates and early stopping to avoid overfitting on small subsets.


\subsection{Variational Quantum Circuit Classifier}
The VQC is implemented with 4 qubits and evaluated in simulation. Input features are reduced to 4 dimensions via PCA (95\% variance retained) and rescaled to match rotation gate domains. We use a ZZFeatureMap for data encoding and a shallow RealAmplitudes ansatz (2 layers). Circuit parameters are optimized using COBYLA over a fixed iteration budget. Classification uses an expectation-value decision function thresholded for binary labels.


% =======================
% Results (compact)
% =======================
\section{Results}


\subsection{Accuracy and Runtime}
Table~\ref{tab:main} reports accuracy and runtime on 250 samples per class. EP reaches 86.4\% accuracy with 0.13s inference on the test batch, while the simulated 4-qubit VQC attains 83.0\% accuracy. Classical baselines provide reference points for accuracy and computational cost.


\begin{table}[t]
\centering
\caption{Performance on 250 samples per class (balanced).}
\label{tab:main}
\begin{tabular}{|l|c|c|c|}
\hline
\textbf{Method} & \textbf{Acc.} & \textbf{Train} & \textbf{Infer.} \\
\hline
CNN (baseline) & 98.4\% & 745s & 0.19s \\
Dense NN (baseline) & 92.0\% & 0.47s & 0.001s \\
EP (ours) & 86.4\% & 89.4s & 0.13s \\
VQC (ours, sim.) & 83.0\% & 180s & 1.0s \\
\hline
\end{tabular}
\end{table}


\subsection{Scaling with Dataset Size}
Figure~\ref{fig:acc} summarizes accuracy across dataset scales (50--250 samples/class). EP improves with more data and shows moderate variance under sampling. The VQC exhibits comparatively stable accuracy across scales in simulation, motivating further study on real hardware with noise and shot-based estimation.


\begin{figure}[t]
\centering
\includegraphics[width=0.48\textwidth]{figure1_accuracy_comparison.png}
\caption{Accuracy across dataset sizes for classical baselines, EP, and VQC.}
\label{fig:acc}
\end{figure}


% Optional second figure if you have room:
% \begin{figure}[t]
% \centering
% \includegraphics[width=0.48\textwidth]{figure2_efficiency_analysis.png}
% \caption{Accuracy vs training time trade-offs.}
% \label{fig:eff}
% \end{figure}


% =======================
% Discussion (tight + careful)
% =======================
\section{Discussion}
EP provides a competitive accuracy--efficiency trade-off while avoiding explicit backpropagation, which may be advantageous for hardware-aware learning systems. The simulated VQC achieves comparable performance despite operating with only four qubits and a shallow circuit, suggesting that quantum feature encodings can be effective on compact representations of microscopy images.


Limitations include: (i) the VQC evaluation is simulation-based and does not capture device noise, (ii) feature engineering constrains the representational capacity compared to end-to-end CNNs, and (iii) the study focuses on binary AML vs. healthy classification rather than a full multi-class hematology taxonomy. Future work will evaluate shot-based sampling, noise-aware training, error mitigation, and hybrid models that combine classical feature extractors with quantum classifiers, as well as benchmarking on additional clinical datasets.


% =======================
% Conclusion (short)
% =======================
\section{Conclusion}
We benchmarked equilibrium propagation and a 4-qubit variational quantum classifier for AML detection from blood cell microscopy images. EP achieved up to 86.4\% accuracy with efficient inference, while the simulated VQC achieved 83.0\% accuracy with stable behavior across dataset scales. These results provide a practical benchmark for QML and quantum-inspired learning algorithms in medical image analysis and motivate follow-on studies on noisy quantum hardware.


% =======================
% Acknowledgments (REMOVE for double-blind)
% Keep this only for camera-ready.
% Also: IEEE requires disclosure if AI-generated text was used.
% =======================
% \section*{Acknowledgments}
% The authors thank ... (dataset providers, funding, etc.)
% AI disclosure example (edit to match your usage):
% Portions of Sections I--V were edited with assistance from ChatGPT (OpenAI) for language clarity; technical content, experiments, and conclusions are the authors' own.


% =======================
% References (trim to 8–10)
% =======================
\begin{thebibliography}{00}


\bibitem{LeCun2015}
Y.~LeCun, Y.~Bengio, and G.~Hinton, ``Deep learning,'' \emph{Nature}, vol.~521, no.~7553, pp.~436--444, 2015.


\bibitem{Benedetti2019}
M.~Benedetti, E.~Lloyd, S.~Sack, and M.~Fiorentini, ``Parameterized quantum circuits as machine learning models,'' \emph{Quantum Science and Technology}, vol.~4, no.~4, p.~043001, 2019.


\bibitem{Farhi2018}
E.~Farhi and H.~Neven, ``Classification with quantum neural networks on near-term processors,'' arXiv:1802.06002, 2018.


\bibitem{Havlicek2019}
V.~Havl{\'\i}{\v{c}}ek \emph{et al.}, ``Supervised learning with quantum-enhanced feature spaces,'' \emph{Nature}, vol.~567, no.~7747, pp.~209--212, 2019.


\bibitem{Scellier2017}
B.~Scellier and Y.~Bengio, ``Equilibrium propagation: Bridging the gap between energy-based models and backpropagation,'' \emph{Frontiers in Computational Neuroscience}, vol.~11, p.~24, 2017.


\bibitem{Matek2019}
C.~Matek, S.~Schwarz, C.~Marr, and K.~Spiekermann, ``A single-cell morphological dataset of leukocytes from AML patients and non-malignant controls,'' \emph{The Cancer Imaging Archive}, 2019, doi:10.7937/tcia.2019.36f5o9ld.


\bibitem{Kandala2017}
A.~Kandala \emph{et al.}, ``Hardware-efficient variational quantum eigensolver for small molecules and quantum magnets,'' \emph{Nature}, vol.~549, no.~7671, pp.~242--246, 2017.


\bibitem{Davies2018}
M.~Davies \emph{et al.}, ``Loihi: A neuromorphic manycore processor with on-chip learning,'' \emph{IEEE Micro}, vol.~38, no.~1, pp.~82--99, 2018.


\end{thebibliography}


\end{document}